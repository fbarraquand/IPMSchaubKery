\documentclass[12pt,a4paper]{article}
\title{IMPs Chapitre 3}
\date{\today}
\author{Tewann Beauchard}
\usepackage[utf8]{inputenc}
\usepackage{amsmath}
\usepackage{amsfonts}
\usepackage{amssymb}
\usepackage{mathpazo}
\usepackage[T1]{fontenc}
\usepackage[french]{babel}

\begin{document}

\begin{center}
\section*{Objectifs du stage}
\vspace*{1cm}
Tewann Beauchard \\
\vspace*{1cm}
May, 2022
\end{center}
\vspace*{1cm}

Lecture du livre INTEGRATED POPULATION MODELS, Theory and Ecological Applications with R and JAGS écrit par Michael Schaub et Marc Kéry (Chapitres 3, 5 et 6): \\
- Répétabilité des résultats\\
- Reproduction des boxplots des moyennes et des écart-type des posterior (page 250 du livre) en partant de deux hypothèses : (1)Estimation des différents paramètres démographiques (productivité,taux de croissance et de survie...) à partir des données de comptage sans dissocier les classes d'âges au sein de ces dernières ($n(t)$).(2)Estimation des différents paramètres démographiques (productivité,taux de croissance et de survie...) à partir des données de comptage en dissociant les classes d'âges au sein de ces dernières ($n_{a}(t)$ et $n_{j}(t)$).


\end{document}
