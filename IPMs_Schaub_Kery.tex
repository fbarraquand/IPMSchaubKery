\documentclass[12pt,a4paper]{article}
\title{IMPs Chapitre 3}
\date{\today}
\author{Tewann Beauchard}
\usepackage[utf8]{inputenc}
\usepackage{amsmath}
\usepackage{amsfonts}
\usepackage{amssymb}
\usepackage{mathpazo}
\usepackage[T1]{fontenc}
\usepackage[french]{babel}

\begin{document}

\begin{center}
\section*{IMPs Schaub et Kéry}
\vspace*{1cm}
Tewann Beauchard \\
\vspace*{1cm}
May, 2022
\end{center}
\vspace*{1cm}

On s'intéresse, ici, au nouveau livre de M.Schaub et M.Kéry qui porte sur les modèles de population intégrés (IMPs). De par sa parution récente, Décembre 2021, il est important de s'interroger sur le fonctionnement et la réplication de leurs modèles. Pour se faire, nous allons nous concentrer en particulier sur les chapitres 3 ("Introduction to stage-structured population models"), 5 ("Introduction to integrated population models") et finalement sur le chapitre 6 ("Benefits of integrated population modeling"). Ainsi, nous pourrons vérifier la reproductibilité des résultats, nous intéresser à l'estimation de $n_{a}(t)$ et de $n_{j}(t)$ qui correspondent aux tailles de populations juvéniles et adultes à l'instant t. Sachant que la taille de la population totale à l'instant t, soit $n(t)$, peut s'écrire : $$n(t)= n_{a}(t) + n_{j}(t)$$L'objectif étant de savoir si la prise en compte de ces deux paramètres, donc de dissocier la taille de population totale par les tailles en fonction des classes d'âges, apportent un réel bénéfice mettant de côté l'utilisation directe de $n(t)$.

\section{Introduction aux modèles de populations structurées en classes d'âges}
En partant du postulat que la taille d'une population à l'instant t est régie par la survie des individus, le recrutement (équivalent quantifiable : le nombre de naissance), l'immigration et l'émigration (tout cela à l'instant t), on se retrouve dans la situation suivante : 
\begin{equation}
N_{t}=S_{t}+R_{t}+I_{t}-E_{t}
\end{equation}

Si l'on décide de s'intéresser aux modèles de population structurés en classes d'âges, il est primordial de prendre en compte le taux de croissance de la population (noté $\lambda_{t}$). C'est même la principale utilitée de ce type de modèle. De plus, ne pas négliger la densité dépendance, l'ajout de la stochasticité (variabilité) et les erreurs d'observations est tout aussi important pour obtenir une meilleure compréhension de la dynamique de la population étudiée.
\begin{equation}
\lambda_{t}=\dfrac{N_{t}}{N_{t-1}}=\dfrac{S_{t}}{N_{t-1}}+\dfrac{R_{t}}{N_{t-1}}+\dfrac{I_{t}}{N_{t-1}}-\dfrac{E_{t}}{N_{t-1}}
\end{equation}
Avec $S_{t}$ la survie, $R_{t}$ le recrutement, $I_{t}$ l'immigration et $E_{t}$ l'émigration à l'instant t.

On s'intéresse dès lors aux modèles structurés en classes d'âges et de stades. Pour le pré-breeding sur les classes d'âges, on a :
\begin{equation}
\begin{aligned}
N_{1y, t+1}=N_{1y, t}f_{1, t}s_{j, t}+N_{2y, t}f_{a, t}s_{j, t}+N_{3y, t}f_{a, t}s_{j, t}+N_{4y, t}f_{a, t}s_{j, t} \\
N_{2y, t+1}=N_{1y, t}s_{a, t} \\
N_{3y, t+1}=N_{2y, t}s_{a, t} \\
N_{4y, t+1}=N_{3y, t}s_{a, t}
\end{aligned}
\end{equation}
Sachant que $s_{j,t}$ désigne la survie des juvéniles de t à t+1 ; $s_{a, t}$ la survie des adultes de t à t+1 ; $f_{1, t}$ la productivité des femelles âgées d'une année à l'instant t ; $f_{a, t}$ la productivité des femelles plus âgées ("adultes") à l'instant t.\vspace*{0.5cm}\\
En ce qui concerne les stades :
\begin{equation}
\begin{aligned}
N_{1y, t+1}=N_{1y, t}f_{1, t}s_{j, t}+N_{ad, t}f_{a, t}s_{j, t} \\
N_{ad, t+1}=N_{1y, t}s_{a, t}+N_{ad, t}s_{a, t}
\end{aligned}
\end{equation}
Pour le post-breeding sur les classes d'âges :
\begin{equation}
\begin{aligned}
N_{juv, t+1}=N_{juv, t}f_{1, t}s_{j, t}+N_{1y, t}f_{a, t}s_{a, t}+N_{2y, t}f_{a, t}s_{a, t}+N_{3y, t}f_{a, t}s_{a, t} \\
N_{1y, t+1}=N_{juv, t}s_{a, t} \\
N_{2y, t+1}=N_{2y, t}s_{a, t} \\
N_{3y, t+1}=N_{3y, t}s_{a, t}
\end{aligned}
\end{equation}
En ce qui concerne les stades :
\begin{equation}
\begin{aligned}
N_{juv, t+1}=N_{juv, t}f_{1, t}s_{j, t}+N_{ad, t}f_{a, t+1}s_{j, t} \\
N_{ad, t+1}=N_{juv, t}s_{a, t}+N_{ad, t}s_{a, t}
\end{aligned}
\end{equation}

Le taux de reproduction net, correspond au nombre de juvénile par lequel un nouveau né peut être remplacé s'il meurt. Là où le taux de croissance $\lambda$ est basé sur la croissance de la population par intervalle de projection, $R_0$, lui, s'exprime en fonction la croissance de la population par génération.
\begin{equation}
R_{0}=\sum_{i=1}^{\infty}l_{i}f_{i}
\end{equation}
Où $l_i$ est la probabilité de survie de la naissance au stade \textit{i} et $f_i$ est la productivité à la classe du stade \textit{i}.
Il en découle le \textit{generation time} (GT) correspondant à : 
\begin{equation}
GT=\dfrac{log(R_0)}{log(\lambda)}
\end{equation}

Nous pouvons ajouter à cela le calcul de la sensibilité (notée $S(a_{i, j})$) et de l'élasticité (notée $E(a_{i, j})$) et permettent de faire des analyses de perturbation en affichage décimal ou en pourcentage respectivement :
\begin{equation}
S(a_{i, j})=\dfrac{\partial\lambda}{\partial a_{i, j}}
\end{equation}
\begin{equation}
E(a_{i, j})=\dfrac{\partial\lambda}{\partial a_{i, j}} \dfrac{a_{i, j}}{\lambda}
\end{equation}

Dans le cas d'un modèle à deux classes d'âges, le taux de croissance est calculé de la manière suivante : 
\begin{equation}
\lambda_{t}=\dfrac{N_{1, t+1}+N_{2, t+1}}{N_{1, t}+N_{2, t}}
\end{equation}
Soit à l'échelle logarithmique :
\begin{equation}
r_{s}=\dfrac{1}{T-u}\sum_{t=u}^{T}r_t
\end{equation}
Où $r_s$ est équivalent à $log(\lambda_s)$.\\La moyenne post-burn-in du taux annuel de croissance est une estimation du taux stochastique de croissance de la population.  
\\
Maintenant, si nous portons notre intérêt sur l'analyse du modèle de matrice de population sans la stochasticité et l'incertitude des paramètres, il faut reformuler les équations du modèle structuré en classes d'âges :
\begin{equation}
N_{1, t+1}=N_{1, t}f_{1, t}s_{j, t}+N_{2, t}f_{a, t}s_{j, t}
\end{equation}
Ce qui nous amène à 
\begin{equation}
N_{2, t+1}=N_{1, t}s_{a, t}+N_{2, t}s_{a, t}=(N_{1, t}+N_{2, t})s_{a, t}
\end{equation}
Ainsi, le taux de croissance s'exprime de la manière suivante :
\begin{equation}
\lambda=\dfrac{N_{1, T+1}+N_{2, T+1}}{N_{1, T}+N_{2, T}}=\dfrac{N_{1, T+1}}{N_{1, T}}=\dfrac{N_{2, T+1}}{N_{2, T}}
\end{equation}
Le noyau de la plupart des IPMs est un modèle de population structuré en stades permettant le lien entre la démographie des individus et la trajectoire de la population.

\section{Introduction aux modèles intégrés de population}
Pour réussir à écrire un IPM, nous passons par plusieurs étapes. La première étant d'ajouter des données de capture-recapture au sein d'un modèle matriciel de population. Ce dernier permet de modéliser la survie des individus aux stades juvénile et adulte, ainsi que la valeur asymptotique du taux de croissance de la population (noté $\lambda$). Ensuite, nous pouvons inclure les données de productivité des individus par le biais d'un modèle de régression de Poisson. À chaque paramètre démographique est associé un prior, ici suivant une loi uniforme. Cependant, nous n'en sommes pas encore à l'écriture d'un IPM. En effet, pour passer de notre analyse jointe actuelle à un IPM, il faut inclure les données de comptage de la population par le biais d'un modèle espace d'état. Ainsi, $\lambda$ est influencé, en partie, par toutes les sources additionnelles de données. En effet, les estimations de certains paramètres sont l'addition de plusieurs sources de données et sont ensuite partagés dans les différents modèles, traduisant ainsi le caractère intégrant du modèle.
Maintenant, dans un but d'inclusion d'une stochasticité environnementale et démographique, on décide de suivre 3 étapes distinctes :

- Développement d'un modèle qui fait le lien entre les taux démographiques et le(s) taille(s) de population.\\

- Formulation de la vraisemblance, séparément, pour chaque jeu de données disponibles (sous-modèle). \\

- Formulation de la vraisemblance jointe ainsi que son analyse pour l'inférence.\\

En suivant ces étapes, nous avons comme point de départ une étude pré-breeding avec 2 classes d'âges ("1y" et "ad") :
\begin{equation}
\begin{aligned}
N_{1y, t+1}=N_{1y, t}\dfrac{f_1}{2}s_{j}+N_{ad, t}\dfrac{f_a}{2}s_{j}
\\
N_{ad, t+1}=(N_{1y, t}+N_{ad, t})s_{a}
\end{aligned}
\end{equation}
Nous avons deux options pour modéliser la variabilité temporelle dans les taux démographiques, un effet temps fixé ou aléatoire avec ou sans corrélation parmi les taux démographiques :
\begin{equation}
\begin{aligned}
N_{1y, t+1} \sim Poisson(N_{1y, t}\dfrac{f_{1,t}}{2}s_{j,t}+N_{ad, t}\dfrac{f_{a,t}}{2}s_{j,t}\\
N_{ad, t+1} \sim Binomial(s_{a,t},N_{1y, t}+N_{ad, t})
\end{aligned}
\end{equation}
Le choix se fera en fonction du code JAGS.
\\
En même temps, pour éviter le double comptage ou une détection imparfaite, on applique au données de comptage une distribution normale : 
\begin{equation}
C_{t} \sim Normal(N_{1y, t}+N_{ad, t},\sigma^{2})
\end{equation}
Sachant que $\sigma^{2}$ correspond à l'erreur constante résiduelle, contenant le manque d'ajustement du modèle de processus et de l'erreur d'observation.\\
Maintenant, on s'intéresse à la vraisemblance du modèle espace d'état (notée $L_ss$) :
\begin{equation}
L_{ss}(N, s_{j}, s_{a}, f_{1}, f_{a}, \sigma^{2}|C)= L_{I}(N_{1})\times L_{S}(N_{2,....T}, s_{j}, s_{a}, f_{1}, f_{a})\times L_{O}(N,\sigma^{2}|C)
\end{equation}
Avec $L_I$ étant à la vraisemblance de la taille de la population à l'occasion 1, $L_S$ étant la vraisemblance du modèle espace d'état et $L_O$ étant la vraisemblance du processus d'observation.\\
Nous portons maintenant notre intérêt sur les données de capture-recapture, suivant un modèle CJS dans notre IPM. Le point départ correspond à la modélisation des probabilités de survie en fonction de l'âge avec des effets aléatoires temporels et non corrélés : 
\begin{equation}
\begin{aligned}
logit(s_{j,t})\sim Normal(\mu_{s_j},\sigma^2_{s_j})
\\
logit(s_{a,t})\sim Normal(\mu_{s_a},\sigma^2_{s_a})
\end{aligned}
\end{equation}
Comme nous assumons que la probabilité de recapture est constante à travers le temps ($p_{t}=\bar{p}$) et que les données sont résumées dans le m-array (noté \textit{m}), alors nous pouvons écrire $L_{CJS}(s_{j}, s_{a}, p|\textit{m})$.\\
En ce qui concerne les données sur la productivité suivant une régression de Poisson dans notre IPM. Le point départ correspond à la modélisation des probabilités de survie en fonction de l'âge avec des effets aléatoires temporels et non corrélés : 
\begin{equation}
\begin{aligned}
\log(f_{1,t})\sim Normal(\mu_{f_1},\sigma^2_{f_1})\\
\log(f_{a,t})\sim Normal(\mu_{f_a},\sigma^2_{f_a})
\end{aligned}
\end{equation}
La vraisemblance du modèle de régression de Poisson est symboliquement $L_{p}(f_{1}, f_{a}|J)$, avec J étant une variable catégorielle renseignant sur la classe d'âge de la mère, en plus d'être la donnée de productivité.
\\
Dès lors, il ne reste plus qu'une étape. Elle consiste en la formulation de la vraisemblance jointe, formant la vraisemblance pour l'ensemble de l'IPM. Elle est le produit de chaque vraisemblance pour chaque jeux de données individuelles discutés auparavant :
\begin{equation}
\begin{aligned}
L_{IPM}(N, s_{j}, s_{a}, p, f_{1}, f_{a}, \sigma^{2}|\textit{m},J,C)= \\
L_{I}(N_{1})\times L_{S}(N_{2,....T}, s_{j}, s_{a}, f_{1}, f_{a})\times L_{O}(N,\sigma^{2}|C)\times L_{CJS}(s_{j}, s_{a}, p|\textit{m})\times L_{p}(f_{1}, f_{a}|J)
\end{aligned}
\end{equation}\vspace*{0,5cm}

Si, maintenant, on s'intéresse à l'ajout de données simulées dans notre IPM, il faut vérifier notre hypothèse d'indépendance des données. Ainsi, un individu présent dans un jeu de données ne doit pas être présent aussi ailleurs. Pour cela, on divise notre population en un nombre de parties équivalent au nombre de jeux de données que l'on veut simuler (comptage de la population, capture-recapture et les données de productivité).

\section{Les bénéfices de la modélisation intégrée de population}
Nous avons vu jusqu'ici comment développer un IPM et son utilité pour analyser des jeux de données au niveau individuel et populationnel. Maintenant, à quel point l'intégration de toutes ces données est au service du résultat. Pour cela, nous allons nous intéresser à quatre bénéfices important apportés par l'utilisation d'un IMP : \\
- Augmenter la précision de l'estimation des paramètres.\\
- L'estimation des paramètres pour lesquels il n'y a pas de données explicites est rendue possible.\\
- L'estimation du processus de covariation entre deux, ou plus, taux démographiques est rendue possible.\\
- L'estimation de la structure de la population et son changement à travers le temps est rendue possible.\vspace*{0,5cm}\\

En ce qui concerne la précision de l'estimation des paramètres, il est logique de penser que si un paramètre est présent dans au moins deux vraisemblance de deux modèles différents, son estimation n'en sera que plus précise.
Il en résulte, après comparaison une à une entre un IPM et les modèles le composant individuellement, que pour un paramètre présent dans tous ces modèles, la mise en commun des informations (avec un IPM) contribue largement à augmenter la précision de l'estimation du paramètre en question. De plus, cette précision augmenterai pour de petites taille d'échantillon (n) (i.e., le Coefficient de Variation a diminué). (documentation : Besbeas et al.,2002; Brooks et al., 2004; Abadi et al., 2010a; Eacket al., 2017)\\
Afin de réduire les biais, il est possible aussi d'ajouter du poids à nos informations, si jamais un doute subsiste sur un jeu de données. Pour connaître le poids de nos données, une cross-validation est la technique adéquate.\\
Une des principales utilité des IPMs est d'estimer des "paramètres cachés" (Tavecchia et al., 2009), notamment la survie lorsque les données de capture-recapture ne sont pas disponibles. En effet, les données de comptage de la population, par vraisemblance jointe, vont servir de support d'information pour la survie. Dans le cas précis du modèle testé par Schaub et Kéry, l'estimation correcte de tous les paramètres est faisable avec au minimum les données de capture-recapture et de comptage. Seule, la productivité n'est pas estimable sans de véritable données de productivité. La tolérance, en terme d'estimations et de précision, serait d'un seul paramètre caché à estimer par l'IPM.




\end{document}
