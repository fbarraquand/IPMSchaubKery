\documentclass[12pt,a4paper]{article}
\title{IMPs Chapitre 3}
\date{\today}
\author{Tewann Beauchard}
\usepackage[utf8]{inputenc}
\usepackage{amsmath}
\usepackage{amsfonts}
\usepackage{amssymb}
\usepackage{mathpazo}
\usepackage[T1]{fontenc}
\usepackage[french]{babel}
\usepackage{graphicx} 

\begin{document}

\begin{center}
\section*{Plan du rapport de stage}
\vspace*{1cm}
Tewann Beauchard \\
\vspace*{1cm}
Août, 2022
\end{center}
\vspace*{1cm}
\section{INTRODUCTION}
- Définition du terme "dynamique des populations" et des facteurs qui régissent cette dynamique.\\

- Par quel(s) moyen(s) ces facteurs peuvent êtres estimés : d'abord il y a la collecte des données (parler des différents types de données qui sont récoltées).\\


Cela amène ensuite à dire que bien souvent il n'est pas possible d'avoir accès à toutes les données, donc indépendamment : \\

- Parler du type de donnée collectée (comptage, productivité, capture-recapture), des paramètres qui peuvent être estimés grâce à ces données (taux de croissance, fécondité, survie) et ainsi du modèle démographique généralement associé.\\


L'analyse se fait souvent de manière indépendante, alors qu'intuitivement on aurait tendance à regrouper l'information pour traduire au mieux la réalité de processus écologiques ayant lieu au sein de la population. Qui plus est, cela nous laisse penser qu'en réunissant les données il y aurait une réduction du biais et une augmentation de la précision.\\

- D'où l'utilisation d'un IPM : description rapide d'un IPM (plus détaillée en matériels et méthodes)\\

- Énumération des bénéfices liés à l'utilisation d'un IPM\\


Malgré leur efficacité, l'exploration de ces modèles est encore réduite : exploration faite en ayant des comptages stade-spécifiques\\


=> Hypothèse/But de l'étude : le bénéfice de dissocier les classes d’âges pour les données de comptage est-il supérieur à la non dissociation au sein d’un modèle intégré de population ?\\


\section{MATÉRIELS ET MÉTHODES}
- Début : Basé sur le code de Schaub et Kéry des analyses ont été menées sur des données simulées : description des deux packages utilisés\\


Concrètement pour écrire le code il faut partir sur le schéma d'écriture d'un IPM :\\

- Premièrement : description des modèles composant un IPM avec les paramètres démographiques associés.\\

- Deuxièmement : écriture du modèle fait le lien entre les taux démographiques de chaque classes d'âges.\\
- Troisièmement : écriture des vraisemblances de chaque modèle individuellement puis, toujours dans un soucis de regroupement de l'information, écriture de la vraisemblance jointe.\\

INCLUSION DU DAG\\
il montre de manière explicite le fonctionnement d'un IPM avec les paramètres démographiques associés aux modèles d'intérêt mais aussi de manière implicite l'utilité d'un IPM à estimer un paramètre caché.\\

Maintenant, si l'on décide de modifier notre IPM pour étudier notre hypothèse de départ tout en voulant tester les limites d'estimation de paramètres cachés, nous partons de la vraisemblance jointe précédemment énoncée : \\

- Déclinaison des vraisemblances pour chaque modèle analysé et représenté dans la partie RÉSULTATS.\\

- Réécriture du modèle démographique faisant le lien entre les taux démographiques des classes d'âges (une fois la dissociation faite).\\


\section{RÉSULTATS}
- Intérêt porté sur la comparaison entre boxplots avec versus sans dissociation des comptages adultes et juvéniles pour rendre compte des performances d'un IPM :\\
GRAPHIQUES DES DIFFÉRENTS BOXPLOTS\\
-le modèle démographique P \& C : dissocier les classes d'âges pour ce modèle permet une meilleure estimation de la survie des juvéniles (en réduisant la valeur de l'écart type) et une estimation plus proche de la valeur utilisée pour la simulation, notamment pour la productivité.\\
-Le modèle démographique CR \& C permet en dissociant les classes d'âges d'estimer plus précisément la valeur du taux de croissance ($\lambda$) \\
- De manière générale : l'estimation est considérablement plus précise pour la survie des adultes.\\

\section{DISCUSSION}
- Reprise des idées de l'INTRODUCTION : \\
Les premières analyses de séries temporelles remontent à longtemps et la plupart du temps seules les données de comptages sont récoltées -> d'où l'intérêt, s'il y a la possibilité, de regrouper les différentes sources de données pour augmenter l'information tirée de l'analyse des données. Rappel du but de l'étude.\\

-> Transition basée sur le modèle espace d'état montrant l'émergence de biais si on utilise une seule source de données ou si l'analyse se fait indépendamment des autres sources de données (elle sert aussi d'exemple pour exprimer la présence de biais au sein des différents modèles).\\

- Explication du choix d'un IPM :\\
-> déclinaison des bénéfices (\textbf{à souligner que je cite les bénéfices concernant l'estimation du processus de covariation entre 2 ou plus taux démographiques et l'estimation de la structure au cours du temps, donc son changement, sans pour autant les reprendre après. Cela est-il nécessaire ?})\\

-> Transition entre les bénéfices d'un IPM en général et les bénéfices de dissocier juvéniles/adultes au sein des données de comptages.\\

-Discussion des graphiques de la partie RÉSULTATS : \\
 - Bénéfice élevé concernant  l'estimation de la survie des adultes\\
 - Tolérance estimée à un seul paramètre caché (moins il y a de paramètres cachés, plus le fait d'estimer en dissociant les classes d'âges est précis et permet une réduction du biais)\\
 - Au delà de cette tolérance il n'y a pas de bénéfice à dissocier les classes d'âges.\\

- Au regard des résultats il faut, si les données sont disponibles : se tourner vers l'utilisation d'un IPM et dissocier les classes d'âges juvéniles/adultes au sein des données de comptages.

\end{document}